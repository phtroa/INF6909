Depuis quelques années le sujet des objets connectés est très étudié. En effet, on leur prévoit un grand avenir, en plus
d'améliorer l'expérience utilisateur au quotidien, ils permettraient d'améliorer la sécurité. Leur aspect connecté permettrait
l'échange d'informations et donc de prévenir leur utiilisateurs d'un danger imminent.

Parmi ces objets on trouve les voitures connectées, ces dernières auraient certaines fonctionnalités destinées à améliorer le
confort et la sécurité routière. Les exemples de telles fonctionnalités ne manquent pas, l'horizon des possibilités s'étend du
simple mécanisme de messagerie à la gestion du traffic en passant par la formation de convois. (ces derniers permettraient une
interaction entre les véhicules voire une automatisation de la conduite). Les applications possibles sont nombreuses et très
étudiées \cite{surveyIntelligentVehicule, fleetConvoyManagent}. Mais, même si les possibilités sont immenses, il ne faut pas
oublier qu'il existera toujours des individus mal intentionnées, des "attaquants", qui tenteront d'exploiter les vulnérabilités
du système. Or dans le cas automobile, les dangers d'une telle attaque sont sans mesures avec ceux auxquels nous sommes habitués.
Si un ordinateur est infecté, la victime subir un dommage financier ou moral, dans le meilleur cas un simple inconfort. Mais dans
un contexte où la victime conduit un véhicule le moindre dysfonctionnent peut mener à un accident pouvant mener à la mort de la
victime et des occupants des véhicules proches.

Ce problème est particulièrement présent dans les réseaux de véhicules ad-hoc où l'abscence d'une infrastructure centralisée rend
difficile la vérification des informations ainsi que l'identification des véhicules malicieux. Pourtant ce type de réseaux est
incontournable. Permettre l'accès un serveur central depuis toues les routes et autoroutes serait extrêment coûteux et prendrait
un temps conséquent. Ainsi, une communocation ad-hoc permettrait, sous réserve qu'une certaine garantie de sécurité existe,
permettrait aux utilisateurs de bénéficier de tous les avantages des voitures connectées à moindre coût, tant pour le particulier,
que pour l'état (ou un quelconque autre organisme qui aurait à sa charge la gestion des routes). En effet, la seule condition pour
qu'un tel réseau puuisse fonctionner est qu'une majorité des véhicules soient équipés des équipements nécessaires.

Aussi, même si sécuriser un réseau ad-hoc est un challenge, le potentiel retour sur investissement suffit à justifier l'effort.
C'est ce constat qui a motivé la construction d'un système de réputation par R. Engoulou \cite{REngoulou}. Système de réputation
dont l'étude a été appronfondie par M. Mallis \cite{MMallis}. Nous allons donc travailler sur un modèle déjà existe afin
d'améliorer ses performances ainsi que la qualité des résultats issus des simulations.

Le modème développé par nos prédécesseurs se base sur la création d'un système de réputation via une modification du protocole
AODV. Le principe de ce système repose sur une modification de l'entête du dit protocole afin d'introduire des variables qui
seront analysées par le système pour produire une note en fonction de laquelle la communication avec le noeud sera acceptée ou
refusée.

\subsection{Mise en contexte}
Avant de présenter notre contribution, il convient de rappeler le fonctionnement du protocole AODV sur lequel se base les travaux
SSVR (ou du moins la partie expérimentale). Cela nous permettra d'introduire le fonctionnement du SSVR, puis de présenter les
enjeux et défis qu'un tel système rencontre dans le milieu des VANETs.

\subsubsection{Le protocole AODV}
Les détails du protocole AODV (Ad hoc On-Demand Distance Vector) sont disponibles dans la RFC 3561 \cite{rfcAODV}. Nous allons
toutefois tenter de présenter les principales caractéristiques de ce protocole. Ce dernier appartient à la catégorie des
protocoles réactifs, c'est-à-dire qu'il va calculer une route perttant à un paquet d'atteindre un autre véhicule seuleemnt si la
nécessité de communiquer avec ce véhicule se présente. De plus, ces routes ne sont maintenues que si elles sont utilisées. Cette
spécificité du protocole permet de limiter la charge du réseau et donc d'améliorer ses performances.

AODV possède trois types principaux de messages:
\begin{itemize}
    \item RREQ (Route Request) ce message permet d'initier la construction d'une route vers une nouvelle destination
    \item RREP  (Route Respond) réponse à un message RREQ si on est la destination ou si l'on a une route vers la destination
    \item RERR  (Route Error) erreur de routage, un lien est devenu invalide par exemple
\end{itemize}
Lors de la construction d'une route le noed émetteur envoie un message \textit{RREQ} via un \textit{broadcast} chaque noeud le
recevant vérifie dans la table de routage s'il possède une route vers la destination si c'est le cas il répond via un
\textit{RREP} sinon il \textit{broadcast} la requête à son tour. Lorsque chaque noeud ayant émis une \textit{RREQ} reçoit une
réponse il ajoute le noeud dont provient la réponse à sa table de routage et désormais il enverra chaque paquet devant être reçu
par la destination à ce noeud qui sera chargé de l'envoyer au noeud qui lui avait répondu déclenchant ainsi lenvoie du
\textit{RREQ}. ce procédé continuera jusqu'à ce que le message atteigneale noued désigné. On soulignera que un message de demande
de route est envoyé uniquement si le noued émetteur n'a pas de route ou si celle-ci n'est plus fonctionnelle.

Le protocole possède bien entendu d'autres subtilités, on pensera par exemple à l'utilisation de numéro de séquence pour éviter la
formation de boucle ou encore à une optimisation basée sur le nombre de sauts pour obtenir des routes les plus courtes possibles,
mais la connaissances de ces dernières n'est nécesaire pour la compréhension du SSVR.


\subsubsection{Système de sécurisation des VANETs par Réputation}
